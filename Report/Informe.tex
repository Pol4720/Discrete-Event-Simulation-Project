\documentclass[a4paper, 12pt]{article}
\usepackage[left=2.5cm, right=2.5cm, top=3cm, bottom=3cm]{geometry}
\usepackage[spanish]{babel}
\usepackage{amsmath}
\usepackage{graphicx}
\usepackage{color}
\usepackage{xcolor}
\usepackage[utf8x]{inputenc}
\usepackage[T1]{fontenc}
\usepackage{listings}
\lstdefinelanguage{JavaScript}{
  keywords={break, case, catch, continue, debugger, default, delete, do, else, false, finally, for, function, if, in, instanceof, new, null, return, switch, this, throw, true, try, typeof, var, void, while, with},
  morecomment=[l]{//},
  morecomment=[s]{/*}{*/},
  morestring=[b]',
  morestring=[b]"
}
\usepackage{tikz}
\usetikzlibrary{shapes,arrows,positioning}
\usepackage{fancyhdr}
\usepackage{titlesec}
\usepackage{background}
\usepackage[hidelinks]{hyperref}
\usepackage{float}

\definecolor{colorgreen}{rgb}{0,0.6,0}
\definecolor{colorgray}{rgb}{0.5,0.5,0.5}
\definecolor{colorpurple}{rgb}{0.58,0,0.82}
\definecolor{colorback}{RGB}{255,255,204}
\definecolor{colorbackground}{RGB}{200,200,221}
\definecolor{bordercolor}{RGB}{0,0,128}

% Definiendo el estilo de las porciones de código
\lstset{
backgroundcolor=\color{colorbackground},
commentstyle=\color{colorgreen},
keywordstyle=\color{colorpurple},
numberstyle=\tiny\color{colorgray},
stringstyle=\color{colorgreen},
basicstyle=\ttfamily\footnotesize,
breakatwhitespace=false,
breaklines=true,
captionpos=b,
keepspaces=true,
numbers=left,
showspaces=false,
showstringspaces=false,
showtabs=false,
tabsize=2,
frame=single,
framesep=2pt,
rulecolor=\color{black},
framerule=1pt
}

% Configuración de encabezado y pie de página
\setlength{\headheight}{15.04742pt}
\addtolength{\topmargin}{-3.04742pt}
\pagestyle{fancy}
\fancyhf{}
\fancyhead[L]{\leftmark}
\fancyhead[R]{\thepage}
\fancyfoot[C]{\textit{Universidad de La Habana - Facultad de Matemática y Computación}}

% Configuración de títulos
\titleformat{\section}
  {\normalfont\Large\bfseries}{\thesection}{1em}{}
\titleformat{\subsection}
  {\normalfont\large\bfseries}{\thesubsection}{1em}{}

% Configuración de fondo de página
\backgroundsetup{
  scale=1,
  color=bordercolor,
  opacity=0.3,
  angle=0,
  position=current page.south,
  vshift=10cm,
  hshift=0cm,
  contents={%
    \begin{tikzpicture}[remember picture,overlay]
      \draw[bordercolor,ultra thick] (current page.south west) rectangle (current page.north east);
    \end{tikzpicture}
  }
}
%sl23

\begin{document}
\graphicspath{{./}}

\begin{titlepage}
    \centering
    \vspace*{2cm}
    {\huge\bfseries Informe\\[0.4cm]}
    {\LARGE Un Problema de Reparación \\}
    \vspace*{2cm}
    \includegraphics[width=0.2\textwidth, height=0.2\textheight]{Images/Presentacion.png}\\[0.5cm]
   
    {\Large \textbf{Richard Alejandro Matos Arderí}\\[0.5cm]}
    {\Large Grupo 311, Ciencia de la Computación\\[0.5cm]}
    {\Large Facultad de Matemática y Computación\\[0.5cm]}
    {\Large Universidad de La Habana\\[0.5cm]}
    \vfill
    \includegraphics[width=0.2\textwidth, height=0.2\textheight]{Images/MATCOM.jpg}\\[0.5cm]
    {\Large 2025}
\end{titlepage}

\newpage
\tableofcontents
\newpage


\section{Introducción}

\section{Definición del problema}
Un sistema necesita $ n $ máquinas en funcionamiento para operar. Para protegerse contra fallos, se mantienen máquinas adicionales como repuestos. Cuando una máquina falla, es reemplazada inmediatamente por un repuesto y enviada a la instalación de reparación. Esta instalación consta de una sola persona que repara las máquinas una a la vez. Una vez reparada, una máquina se convierte en un repuesto disponible (ver Figura 7.4). Los tiempos de reparación son variables aleatorias independientes con una función de distribución común $ G $. El tiempo de funcionamiento antes de fallar, para cada máquina, es una variable aleatoria independiente con función de distribución $ F $.

El sistema "colapsa" cuando una máquina falla y no hay repuestos disponibles. Suponiendo que inicialmente hay $ n + s $ máquinas funcionales ($ n $ en uso y $ s $ como repuestos), estamos interesados en simular este sistema para aproximar $ E[T] $, donde $ T $ es el tiempo en que el sistema colapsa.

\subsection*{Variables Utilizadas en la Simulación}

    
- \textbf{Variable de tiempo:} $ t $
    
- \textbf{Variable de estado del sistema:} $ r $, el número de máquinas fuera de servicio en el tiempo $ t $.


Se dice que ocurre un ``evento`` cuando:
\begin{enumerate}
    \item Una máquina en funcionamiento falla.
    \item Se completa una reparación.
\end{enumerate}

Para determinar cuándo ocurrirá el próximo evento, necesitamos realizar un seguimiento de los tiempos de fallo de las máquinas en uso y el tiempo de finalización de la reparación actual. Es conveniente almacenar estos tiempos en una lista ordenada:

\[
\text{Lista de eventos: } t_1 \leq t_2 \leq t_3 \leq \cdots \leq t_n, t^*
\]
donde $ t_1, \dots, t_n $ son los tiempos (en orden) en que las $ n $ máquinas en uso fallarán, y $ t^* $ es el tiempo en que la máquina en reparación volverá a estar operativa, o $ t^* = \infty $ si no hay ninguna máquina en reparación.

\subsection*{Inicialización de la Simulación}
\begin{enumerate}
    \item Establecer $ t = r = 0 $, $ t^* = \infty $.
    \item Generar $ X_1, \dots, X_n $, variables aleatorias independientes con distribución $ F $.
    \item Ordenar estos valores y asignar $ t_i $ como el $ i $-ésimo valor más pequeño, $ i = 1, \dots, n $.
    \item Establecer la lista de eventos: $ t_1, \dots, t_n, t^* $.
\end{enumerate}

\subsection*{Actualización del Sistema}
La actualización depende de los siguientes casos:

\subsubsection*{Caso 1: $ t_1 < t^* $}
\begin{enumerate}
    \item Restablecer $ t = t_1 $.
    \item Restablecer $ r = r + 1 $ (otra máquina ha fallado).
    \item Si $ r = s + 1 $, detener la simulación y registrar $ T = t $ (el sistema colapsa).
    \item Si $ r < s + 1 $:
    
        
- Generar una variable aleatoria $ X $ con distribución $ F $ (tiempo de funcionamiento del repuesto).
        
- Reordenar los valores $ t_2, t_3, \dots, t_n, t + X $ y actualizar $ t_i $ como el $ i $-ésimo valor más pequeño.
        
- Si $ r = 1 $, generar una variable aleatoria $ Y $ con distribución $ G $ y restablecer $ t^* = t + Y $.
    
\end{enumerate}

\subsubsection*{Caso 2: $ t^* \leq t_1 $}
\begin{enumerate}
    \item Restablecer $ t = t^* $.
    \item Restablecer $ r = r - 1 $.
    \item Si $ r > 0 $, generar una variable aleatoria $ Y $ con distribución $ G $ y restablecer $ t^* = t + Y $.
    \item Si $ r = 0 $, establecer $ t^* = \infty $.
\end{enumerate}

\subsection*{Resultados de la Simulación}
Cada vez que el sistema colapsa ($ r = s + 1 $), decimos que se completa una ejecución. La salida de la ejecución es el tiempo de colapso $ T $. Realizamos $ k $ ejecuciones, con salidas sucesivas $ T_1, \dots, T_k $. Estas variables son independientes y representan tiempos de colapso. Su promedio,
\[
\frac{1}{k} \sum_{i=1}^k T_i,
\]
es una estimación de $ E[T] $, el tiempo medio de colapso. El método para determinar el valor de $ k $ se discute en el Capítulo 8, que presenta técnicas para analizar estadísticamente los resultados de las simulaciones.


\end{document}












































